\documentclass[french]{article}
\usepackage{graphicx}

\usepackage[utf8]{inputenc}
\usepackage[T1]{fontenc}
\usepackage{babel}
\usepackage{xcolor}

\newcommand{\elodie}{\hfill{\color{red}{Élodie}}}
\newcommand{\valentin}{\hfill{\color{blue}{Valentin}}}
\newcommand{\alexis}{\hfill{\color{green}{Alexis}}}

\begin{document}

\title{Liste des fonctions à coder}
%\author{Author's Name}

\maketitle

Il s'agira de commenter les actions les plus importantes effectuées dans les fonctions, utiliser les variables définies dans la classe et de vérifier la bonne exécution des fonctions.

\section{Fonctions dans le dossier Tools}
\subsection{Fichier border}
\textbf{MakeListOfIndexPoints} :\alexis
\begin{itemize}
\item Un pointeur vers un objet Mesh,
\item et un pointeur vers une fonction f.
\end{itemize}
Cette fonction trouve les noeuds qui sont supposés définir la frontière $\Omega$ et elle les rajoute au maillage. Elle retourne ensuite les indices des points sur cette frontière.

\subsection{Fichier errors}
\textbf{GetErrorL1 :}\elodie
\begin{itemize}
\item Un pointeur vers un objet Mesh,
\item un vecteur de la solution analytique sur le maillage,
\item et un vecteur de la solution numérique sur le maillage.
\end{itemize}
Cette fonction calcule $\|u_{ana} - u_{num}\|_{l^1}$.\\

\textbf{GetErrorLinf :}\elodie
\begin{itemize}
\item Un pointeur vers un objet Mesh,
\item un vecteur de la solution analytique sur le maillage,
\item et un vecteur de la solution numérique sur le maillage.
\end{itemize}
Cette fonction calcule $\|u_{ana} - u_{num}\|_{l^\infty}$.\\

\textbf{GetErrorRela :}\alexis
\begin{itemize}
\item Un pointeur vers un objet Mesh,
\item un vecteur de la solution analytique sur le maillage,
\item et un vecteur de la solution numérique sur le maillage.
\end{itemize}
Cette fonction calcule $\frac{\|u_{ana} - u_{num}\|_{l^2}}{\|u_{ana}\|_{l^2}}$ (c'est un genre de pourcentage).\\

\textbf{GetErrorAbs :}\alexis
\begin{itemize}
\item Un pointeur vers un objet Mesh,
\item un vecteur de la solution analytique sur le maillage,
\item et un vecteur de la solution numérique sur le maillage.
\end{itemize}
Cette fonction calcule $v\:=|u_{ana}-u_{num}|$ (valeur absolue).

\subsection{Fichier funtovec}
\textbf{FuntToVec :}\elodie
\begin{itemize}
\item Un pointeur vers un objet Mesh,
\item un pointeur vers une fonction f dépendante des coordonnées spatiales et temporelles,
\item et d'un t (optionnel, par défaut il vaut 0).
\end{itemize}
Cette fonction retourne un vecteur sur le maillage (de taille nombre de points) et l'initialise pour le temps dt en chaque points du maillage.\\

\textbf{FuntToVec :}\elodie
\begin{itemize}
\item Un pointeur vers un objet Mesh,
\item et d'une valeur réelle.
\end{itemize}
Cette fonction retourne un vecteur sur le maillage (de taille nombre de points) et l'initialise à la valeur passée en argument.

\subsection{Fichier impose}
\textbf{ImposeDirichlet :}\alexis
\begin{itemize}
\item Un pointeur vers un objet Mesh,
\item un pointeur vers une sparsematrix d'Eigen,
\item un pointeur vers une fonction g dépendant des coordonnées spatiales et temporelle,
\item un vecteur d'entiers regroupant les indices des points comme définissant la frontière de $\Omega$,
\item et d'un t (optionnel, par défaut il vaut 0).
\end{itemize}
Cette fonction modifie la sparsematrix pour imposer la conditions de Dirichlet et elle retourne le vecteur des conditions imposées (voir calculs).\\

\textbf{ImposeNeumann :}\valentin
\begin{itemize}
\item Un pointeur vers un objet Mesh,
\item un pointeur vers une sparsematrix d'Eigen,
\item un pointeur vers une fonction g dépendant des coordonnées spatiales et temporelle,
\item un vecteur d'entiers regroupant les indices des points comme définissant la frontière de $\Omega$,
\item et d'un t (optionnel, par défaut il vaut 0).
\end{itemize}
Cette fonction modifie la sparsematrix pour imposer la conditions de Neumann et elle retourne le vecteur des conditions imposées (voir calculs).

\subsection{Fichier matrix}

\textbf{BuildMatrixLaplaceEquation :}\elodie
\begin{itemize}
\item Un pointeur vers un objet Mesh.
\end{itemize}
Cette fonction retourne une sparsematrix du problème de Laplace sans préoccupation des conditions aux limites.

\textbf{BuildMatrixHeatEquation :}\alexis
\begin{itemize}
\item Un pointeur vers un objet Mesh,
\item un réel dt,
\item et un réel qui est le coefficient de l'équation.
\end{itemize}
Cette fonction retourne une sparsematrix du problème de la chaleur sans préoccupation des conditions aux limites.

\subsection{Fichier solver}

\textbf{Solve :}\elodie
\begin{itemize}
\item Une sparsematrix,
\item un vecteur symbolisant à la fois le second membre dans le cas implicite et la valeur au temps précédent dans le cas explicite,
\item et un entier (valant EXPLICIT ou IMPLICIT) pour préciser le type de résolution.
\end{itemize}
Cette fonction résout au choix le problème implicite avec le gradient conjugué d'Eigen ou le simple produit matrice-vecteur dans le cas explicite et elle retourne le vecteur résultant.


\section{Classe Cell dans Data }
\begin{itemize}
\item Constructeur : regarder quelles variables doivent être impérativement initialisées pour éviter les erreurs de l'utilisateur.
\item Destructeur : supprimer les pointeurs qui sont déclarés uniquement au sein de la classe
\item les multiples pointeurs vers les cellules environnantes \valentin
\item calcul du barycentre \valentin
\item list des points définissant la celulle \valentin
\end{itemize}


\section{Classe Mesh dans le dossier Mesh }
\begin{itemize}
\item Je m'en occupe
\end{itemize}

\section{Classe Point dans le dossier Data }
\begin{itemize}
\item Déjà fait
\end{itemize}


\section{Classe Writer dans le dossier Writer}
\begin{itemize}
\item Constructeur : regarder quelles variables doivent être impérativement initialisées pour éviter les erreurs de l'utilisateur.
\item Destructeur : supprimer les pointeurs qui sont déclarés uniquement au sein de la classe
\item SetVector[Numerical, Analytical, ErrorAbs] : affecte en interne
\item WriteBothDomains[On, Off] : affecte en interne
\item WriteNow : écrit un fichier ".vtk" et des fichiers ".dat" \valentin
\end{itemize}
\end{document}